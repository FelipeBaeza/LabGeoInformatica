\documentclass[12pt,a4paper]{article}
\usepackage[utf8]{inputenc}
\usepackage[spanish,es-tabla]{babel}
\usepackage{geometry}
\geometry{margin=2.5cm}
\usepackage{graphicx}
\usepackage{float}
\usepackage{amsmath,amssymb,amsthm}
\usepackage{enumerate}
\usepackage{xcolor}
\usepackage{hyperref}
\usepackage{tcolorbox}
\usepackage{booktabs}
\usepackage{longtable}
\usepackage{array}
\usepackage{multirow}
\usepackage{fancyhdr}
\usepackage{listings}
\usepackage{tikz}
\usetikzlibrary{shapes,arrows,positioning,calc}
\usepackage{fontawesome5}

% Configuración de encabezados
\pagestyle{fancy}
\fancyhf{}
\lhead{Laboratorio Integrador}
\rhead{Geoinformática 2025}
\cfoot{\thepage}

% Comandos personalizados
\newtcolorbox{taskbox}[2][]{
    colback=blue!5!white,
    colframe=blue!75!black,
    title=#2,
    fonttitle=\bfseries,
    #1
}

\newtcolorbox{alertbox}[1][]{
    colback=red!5!white,
    colframe=red!75!black,
    fonttitle=\bfseries,
    #1
}

\newtcolorbox{deliverable}[2][]{
    colback=yellow!5!white,
    colframe=yellow!75!black,
    title=#2,
    fonttitle=\bfseries,
    #1
}

% Configuración de listings para código
\lstset{
    basicstyle=\ttfamily\small,
    breaklines=true,
    frame=single,
    language=Python,
    numbers=left,
    numberstyle=\tiny\color{gray},
    keywordstyle=\color{blue},
    commentstyle=\color{green!60!black},
    stringstyle=\color{red}
}

\title{
    \vspace{-2cm}
    \Large{UNIVERSIDAD DE SANTIAGO DE CHILE} \\
    \large{Facultad de Ingeniería} \\
    \large{Departamento de Ingeniería Informática} \\
    \vspace{1cm}
    \LARGE{\textbf{Sistema de Análisis Geoespacial Integral}} \\
    \Large{\textbf{Isla de Pascua - Rapa Nui}} \\
    \vspace{0.5cm}
    \large{Laboratorio Integrador - Geoinformática}
}

\author{
    Felipe Baeza \\
    \texttt{felipe.baeza.m@usach.cl}
    Catalina López \\
    \texttt{catalina.lopez.a@usach.cl}
}

\date{\today}

\begin{document}

\maketitle
\thispagestyle{empty}
\newpage

\tableofcontents
\newpage

%==============================================================================
\section{Resumen Ejecutivo}
%==============================================================================

Este proyecto desarrolla un \textbf{Sistema de Análisis Geoespacial Integral} para Isla de Pascua (Rapa Nui), abordando el problema de la \textbf{presión demográfica} y la necesidad de monitorear la \textbf{capacidad de carga territorial} establecida por la Ley 21.070. El sistema combina técnicas de análisis espacial, geoestadística y machine learning sobre datos multi-fuente:

\begin{itemize}
    \item \textbf{Infraestructura containerizada} con Docker Compose (PostGIS + Jupyter + Streamlit)
    \item \textbf{Base de datos espacial} PostGIS con 23 capas geográficas vectoriales
    \item \textbf{Pipeline automatizado} de descarga y procesamiento de datos desde OpenStreetMap
    \item \textbf{Análisis exploratorio espacial} (ESDA) con autocorrelación y hot spots
    \item \textbf{Modelos de machine learning} para predicción de densidad edificatoria
    \item \textbf{Aplicación web interactiva} con Streamlit para visualización de resultados
    \item \textbf{Indicadores territoriales} alineados con el Modelo de Capacidad de Carga (MCCIP)
\end{itemize}

El proyecto demuestra cómo la geoinformática puede contribuir a generar \textbf{evidencia territorial} para apoyar la implementación de políticas públicas en un territorio especial con desafíos únicos de sostenibilidad.

%==============================================================================
\section{Introducción}
%==============================================================================

\subsection{Contexto del Problema}

Isla de Pascua (Rapa Nui) es un \textbf{Territorio Especial} de Chile ubicado en el Océano Pacífico, conocido mundialmente por sus moáis y su singular patrimonio cultural. Sin embargo, enfrenta desafíos críticos de planificación territorial reconocidos por la \textbf{Ley 21.070 (2018)}:

\begin{itemize}
    \item \textbf{Crecimiento poblacional acelerado}: 116\% de aumento entre 2002-2017
    \item \textbf{Presión turística}: Sobre 150.000 visitantes anuales pre-pandemia
    \item \textbf{Saturación de infraestructura}: Sistemas de agua, alcantarillado y residuos al límite
    \item \textbf{Expansión urbana descontrolada}: Desbordamiento hacia zonas arqueológicas protegidas
    \item \textbf{Fragilidad ecosistémica}: Acuífero limitado con riesgo de salinización
    \item \textbf{Escasez de herramientas de monitoreo}: Necesidad de indicadores territoriales cuantitativos
\end{itemize}

\subsection{Justificación}

Este proyecto es relevante porque:

\begin{enumerate}
    \item \textbf{Aplicación práctica}: Genera información útil para planificación territorial
    \item \textbf{Metodología replicable}: El pipeline puede adaptarse a otras comunas
    \item \textbf{Tecnología de punta}: Demuestra uso de herramientas modernas (Docker, ML espacial)
    \item \textbf{Datos abiertos}: Todo basado en OpenStreetMap y fuentes públicas
    \item \textbf{Territorio único}: Rapa Nu i presenta características geográficas singulares
\end{enumerate}

\subsection{Objetivos}

\subsubsection{Objetivo General}

Desarrollar un sistema integral de análisis geoespacial que permita caracterizar, analizar y visualizar patrones territoriales de Isla de Pascua mediante técnicas de ciencia de datos espaciales.

\subsubsection{Objetivos Específicos}

\begin{enumerate}
    \item \textbf{OE1}: Implementar infraestructura reproducible con Docker y PostGIS
    \item \textbf{OE2}: Adquirir y procesar datos multi-fuente (vectorial OSM)
    \item \textbf{OE3}: Realizar análisis exploratorio espacial (ESDA) con autocorrelación
    \item \textbf{OE4}: Aplicar técnicas de detección de hot spots (Getis-Ord Gi*)
    \item \textbf{OE5}: Desarrollar modelos predictivos de machine learning espacial
    \item \textbf{OE6}: Crear aplicación web interactiva para exploración de resultados
\end{enumerate}

\subsection{Preguntas de Investigación}

\begin{enumerate}
    \item ¿Cuáles son los patrones de distribución espacial de edificaciones en Rapa Nui?
    \item ¿Existen clusters significativos (hot spots) de densidad urbana?
    \item ¿Qué variables espaciales predicen mejor la densidad edificatoria?
    \item ¿Cómo se relaciona la infraestructura vial con la distribución de amenidades?
    \item ¿Cómo puede la geoinformática contribuir al monitoreo de la capacidad de carga demográfica?
\end{enumerate}

%==============================================================================
\section{Problema Territorial: Presión Demográfica}
%==============================================================================

\subsection{Contexto del Territorio Especial}

Isla de Pascua (Rapa Nui) constituye un \textbf{Territorio Especial} según la Constitución Política de Chile (Art. 126 bis), con características únicas que demandan una gestión territorial diferenciada:

\begin{itemize}
    \item \textbf{Aislamiento extremo}: Ubicada a 3.700 km del continente, es uno de los territorios habitados más remotos del mundo
    \item \textbf{Fragilidad ecológica}: Superficie limitada de 163,6 km² con ecosistemas insulares vulnerables
    \item \textbf{Patrimonio cultural único}: Sitio Patrimonio de la Humanidad UNESCO desde 1995
    \item \textbf{Población creciente}: Incremento sostenido de residentes y turistas que presionan la infraestructura
\end{itemize}

\subsection{Diagnóstico: Sobrecarga Demográfica}

Según el \textbf{Modelo de Capacidad de Carga Demográfica (MCCIP)} desarrollado por el Instituto de Estudios Urbanos UC para SUBDERE, Isla de Pascua enfrenta una crisis de sostenibilidad territorial:

\begin{table}[H]
\centering
\begin{tabular}{ll}
\toprule
\textbf{Factor} & \textbf{Situación Crítica} \\
\midrule
Crecimiento poblacional & Aumento del 116\% entre 2002-2017 (de 3.791 a 8.200 hab.) \\
Flujo turístico & Sobre 150.000 visitantes anuales pre-pandemia \\
Infraestructura & Saturación de agua, alcantarillado y residuos \\
Expansión urbana & Desbordamiento hacia zonas arqueológicas \\
Recursos hídricos & Acuífero limitado con riesgo de salinización \\
\bottomrule
\end{tabular}
\caption{Factores críticos de presión demográfica en Rapa Nui}
\end{table}

\subsection{Marco Legal: Ley 21.070}

La \textbf{Ley 21.070 (2018)} ``Regula el ejercicio de los derechos a residir, permanecer y trasladarse hacia y desde el territorio especial de Isla de Pascua'' establece:

\begin{itemize}
    \item Restricción de permanencia máxima de 30 días para turistas
    \item Requisitos de acreditación para residentes no Rapa Nui
    \item Obligación de establecer la \textbf{capacidad de carga demográfica} (Art. 13)
    \item Creación del \textbf{Índice Pascua (IPA)} como indicador integrado
\end{itemize}

El \textbf{Decreto N° 1120 (2018)} establece la fórmula para calcular la capacidad de carga basada en 21 variables agrupadas en 4 subsistemas:

\begin{enumerate}
    \item \textbf{Subsistema Población}: Dinámica demográfica, migración, turismo
    \item \textbf{Subsistema Medio Ambiente}: Agua, residuos, energía, biodiversidad
    \item \textbf{Subsistema Económico}: Empleo, vivienda, transporte, servicios
    \item \textbf{Subsistema Socio-Cultural}: Patrimonio, identidad Rapa Nui, calidad de vida
\end{enumerate}

\subsection{Contribución de la Geoinformática}

Este proyecto contribuye al monitoreo territorial mediante:

\begin{itemize}
    \item \textbf{Indicadores espaciales}: Densidad de edificaciones, cobertura de servicios
    \item \textbf{Detección de patrones}: Hot spots de concentración urbana
    \item \textbf{Modelos predictivos}: Proyección de expansión territorial
    \item \textbf{Visualización}: Dashboard para tomadores de decisiones
\end{itemize}

%==============================================================================
\section{Área de Estudio}
%==============================================================================

\subsection{Delimitación Geográfica}

\textbf{Nombre}: Isla de Pascua (Rapa Nui) \\
\textbf{Región}: Región de Valparaíso \\
\textbf{Provincia}: Isla de Pascua \\
\textbf{Comuna}: Isla de Pascua \\
\textbf{Coordenadas centrales}: -27.1167° S, -109.3667° W \\
\textbf{Área total}: $\sim$164 km$^2$ \\
\textbf{Población (Censo 2017)}: 7,750 habitantes \\
\textbf{Proyección utilizada}: EPSG:32719 (UTM 19S)

\subsection{Características Territoriales}

\subsubsection{Geomor fología}

Isla de Pascua es de origen volcánico, con tres volcanes principales:
\begin{itemize}
    \item \textbf{Rano Kau} (324 m): Volcán extinto al suroeste
    \item \textbf{Poike} (370 m): Peninsula volcánica al este
    \item \textbf{Maunga Terevaka} (507 m): Punto más alto de la isla
\end{itemize}

\subsubsection{Uso del Suelo}

\begin{itemize}
    \item \textbf{Urbano}: Concentrado en Hanga Roa (oeste)
    \item \textbf{Protegido}: Parque Nacional Rapa Nui (43\% del territorio)
    \item \textbf{Rural}: Praderas y sitios arqueológicos distribuidos
    \item \textbf{Aeroportuario}: Aeropuerto Internacional Mataveri
\end{itemize}

\subsection{Justificación de la Selección}

Isla de Pascua fue seleccionada por:

\begin{enumerate}
    \item \textbf{Tamaño manejable}: Escala apropiada para análisis completo
    \item \textbf{Datos disponibles}: Buena cobertura en OpenStreetMap
    \item \textbf{Interés territorial}: Desafíos reales de planificación
    \item \textbf{Unicidad geográfica}: Caso de estudio singular
    \item \textbf{Complejidad moderada}: Ideal para proyecto académico
\end{enumerate}

%==============================================================================
\section{Datos y Metodología}
%==============================================================================

\subsection{Fuentes de Datos}

\begin{table}[H]
\centering
\begin{tabular}{llll}
\toprule
\textbf{Tipo} & \textbf{Fuente} & \textbf{Elementos} & \textbf{Uso} \\
\midrule
Límite administrativo & OSM & 1 & Base cartográfica \\
Red vial & OSM & 4,139 & Análisis conectividad \\
Edificaciones & OSM & 4,045 & Variable objetivo ML \\
Puntos de interés & OSM & 241 & Features espaciales \\
Áreas verdes/playas & OSM & 12 & Contexto ambiental \\
Transporte & OSM & 1 & Infraestructura \\
\bottomrule
\end{tabular}
\caption{Datasets integrados en PostGIS}
\end{table}

\textbf{Fecha de descarga}: Diciembre 2024 \\
\textbf{Herramienta}: OSMnx (Python) \\
\textbf{Formato original}: GeoJSON \\
\textbf{Sistema de almacenamiento}: PostgreSQL/PostGIS

\subsection{Arquitectura del Sistema}

\subsubsection{Infraestructura Docker}

El proyecto utiliza Docker Compose con 3 servicios:

\begin{verbatim}
services:
  postgis:
    image: postgis/postgis:15-3.3
    ports: ["55432:5432"]
    volumes: [postgres_data, init scripts]
    
  jupyter:
    build: ./docker/jupyter
    ports: ["8888:8888"]
    depends_on: [postgis]
    
  streamlit:
    build: ./docker/streamlit
    ports: ["8501:8501"]
    depends_on: [postgis]
\end{verbatim}

\textbf{Ventajas}:
\begin{itemize}
    \item Reproducibilidad total del ambiente
    \item Aislamiento de dependencias
    \item Fácil despliegue en diferentes máquinas
    \item Persistencia de datos con volúmenes
\end{itemize}

\subsubsection{Base de Datos PostGIS}

\textbf{Versión}: PostgreSQL 15 con PostGIS 3.3 \\
\textbf{Esquema}: \texttt{geoanalisis} \\
\textbf{Extensiones activas}:
\begin{itemize}
    \item \texttt{postgis}: Tipos espaciales y funciones
    \item \texttt{postgis\_topology}: Análisis topológico
\end{itemize}

\textbf{Tablas creadas} (23 capas total):
\begin{itemize}
    \item \texttt{limite\_administrativa}
    \item \texttt{linea\_calles}
    \item \texttt{area\_construcciones}
    \item \texttt{punto\_interes}
    \item \texttt{area\_naturaleza\_playas}
    \item Y 18 capas adicionales de diferentes categorías OSM
\end{itemize}

\subsection{Pipeline de Procesamiento}

\begin{figure}[H]
\centering
\begin{tikzpicture}[
    node distance=1.5cm,
    every node/.style={font=\small},
    box/.style={rectangle, draw, fill=blue!20, text width=3cm, text centered, minimum height=1cm}
]
    \node [box] (osm) {1. Descarga OSM\\(OSMnx)};
    \node [box, below of=osm] (geojson) {2. GeoJSON\\(Exportación)};
    \node [box, below of=geojson] (postgis) {3. PostGIS\\(Carga)};
    \node [box, below of=postgis] (analysis) {4. Análisis\\(Python/SQL)};
    \node [box, below of=analysis] (viz) {5. Visualización\\(Streamlit)};
    
    \draw [->] (osm) -- (geojson);
    \draw [->] (geojson) -- (postgis);
    \draw [->] (postgis) -- (analysis);
    \draw [->] (analysis) -- (viz);
\end{tikzpicture}
\caption{Flujo de datos del proyecto}
\end{figure}

\subsection{Herramientas y Librerías}

\subsubsection{Python}

\textbf{Geoespaciales}:
\begin{itemize}
    \item GeoPandas 0.14.0: Manipulación de datos vectoriales
    \item Shapely 2.0.2: Operaciones geométricas
    \item PyProj 3.6.1: Transformaciones de CRS
    \item OSMnx 1.7.1: Descarga de datos OpenStreetMap
\end{itemize}

\textbf{Análisis Espacial}:
\begin{itemize}
    \item libpysal 4.9.0: Pesos espaciales
    \item esda 2.5.0: Estadística espacial
    \item splot 1.1.5: Visualización ESDA
\end{itemize}

\textbf{Machine Learning}:
\begin{itemize}
    \item scikit-learn 1.3.0: Modelos predictivos
    \item XGBoost 2.0.0: Gradient boosting
\end{itemize}

\textbf{Visualización}:
\begin{itemize}
    \item Matplotlib 3.8.0: Gráficos estáticos
    \item Folium 0.15.0: Mapas interactivos
    \item Streamlit 1.28.0: Dashboard web
\end{itemize}

%==============================================================================
\section{Implementación}
%==============================================================================

\subsection{Parte 1: Ambiente de Desarrollo}

\begin{deliverable}{Entregable 1: Configuración Docker}
\begin{itemize}
    \item[\checkmark] Docker Compose funcional
    \item[\checkmark] PostGIS con extensiones activadas
    \item[\checkmark] Jupyter Lab con kernel geoespacial
    \item[\checkmark] Scripts de inicialización automáticos
    \item[\checkmark] Documentación completa (README.md)
\end{itemize}
\end{deliverable}

La configuración Docker permite levantar todo el ambiente con un solo comando:

\begin{verbatim}
docker compose up -d
\end{verbatim}

Los servicios quedan accesibles en:
\begin{itemize}
    \item \textbf{PostGIS}: localhost:55432
    \item \textbf{Jupyter}: localhost:8888
    \item \textbf{Streamlit}: localhost:8501
\end{itemize}

\subsection{Parte 2: Adquisición de Datos}

\begin{deliverable}{Entregable 2: Dataset Integrado}
\begin{itemize}
    \item[\checkmark] Datos vectoriales de OSM (23 capas)
    \item[\checkmark] Red vial completa (4,139 elementos)
    \item[\checkmark] Edificaciones (4,045 polígonos)
    \item[\checkmark] Puntos de interés (241 amenidades)
    \item[\checkmark] Todo cargado en PostGIS
\end{itemize}
\end{deliverable}

\subsubsection{Script de Descarga}

El archivo \texttt{scripts/01\_download\_osm.py} descarga todos los datos:

\begin{lstlisting}[language=Python]
import osmnx as ox

# Configuracion
place_name = "Isla de Pascua, Chile"

# Descargar limite
boundary = ox.geocode_to_gdf(place_name)
boundary.to_file("data/boundary.geojson")

# Descargar red vial
G = ox.graph_from_place(place_name, network_type='all')
ox.save_graph_geopackage(G, "data/streets.gpkg")

# Descargar edificaciones
buildings = ox.features_from_place(
    place_name, 
    tags={'building': True}
)
buildings.to_file("data/buildings.geojson")

# Descargar amenidades
amenities = ox.features_from_place(
    place_name,
    tags={'amenity': True}
)
amenities.to_file("data/amenities.geojson")
\end{lstlisting}

\subsection{Parte 3: Análisis Exploratorio Espacial}

\begin{deliverable}{Entregable 3: ESDA Completo}
\begin{itemize}
    \item[\checkmark] Estadísticas descriptivas espaciales
    \item[\checkmark] Mapas temáticos (10+ generados)
    \item[\checkmark] Análisis de hot spots (Getis-Ord Gi*)
    \item[\checkmark] Visualizaciones interactivas
\end{itemize}
\end{deliverable}

\subsubsection{Análisis de Densidad Espacial}

Se implementó análisis de densidad mediante grillas de 200m×200m:

\begin{lstlisting}[language=Python]
def create_grid(boundary_gdf, cell_size=200):
    """Crear grilla regular para analisis"""
    minx, miny, maxx, maxy = boundary_gdf.total_bounds
    
    cells = []
    for x in range(int(minx), int(maxx), cell_size):
        for y in range(int(miny), int(maxy), cell_size):
            cell = box(x, y, x + cell_size, y + cell_size)
            cells.append(cell)
    
    grid = gpd.GeoDataFrame(geometry=cells, crs=boundary_gdf.crs)
    return grid[grid.intersects(boundary_gdf.unary_union)]
\end{lstlisting}

\textbf{Optimización crítica}: Se reemplazaron loops O(n*m) por spatial joins vectorizados:

\begin{lstlisting}[language=Python]
# ANTES: Muy lento (minutos)
for idx, cell in grid.iterrows():
    count = len(features[features.within(cell.geometry)])
    grid.loc[idx, 'count'] = count

# DESPUES: Rapido (segundos)
joined = gpd.sjoin(features, grid, predicate='within')
counts = joined.groupby('cell_id').size()
grid = grid.merge(counts, on='cell_id', how='left')
\end{lstlisting}

\subsubsection{Hot Spot Analysis}

Implementación de Getis-Ord Gi*:

\begin{lstlisting}[language=Python]
from libpysal.weights import Queen
from esda.getisord import G_Local

# Crear matriz de pesos espaciales
w = Queen.from_dataframe(grid_analysis)
w.transform = 'r'  # Row standardization

# Calcular Gi*
gi = G_Local(grid_analysis['n_features'].values, w, star=True)
grid_analysis['gi_z'] = gi.Zs
grid_analysis['gi_p'] = gi.p_sim

# Clasificar hotspots
grid_analysis['hotspot_type'] = 'No significativo'
grid_analysis.loc[
    (grid_analysis['gi_z'] > 1.96) & 
    (grid_analysis['gi_p'] < 0.05), 
    'hotspot_type'
] = 'Hot Spot (95%)'
\end{lstlisting}

\subsection{Parte 4: Machine Learning Espacial}

\begin{deliverable}{Entregable 4: Modelo Predictivo}
\begin{itemize}
    \item[\checkmark] Problema: Predicción de densidad edificatoria
    \item[\checkmark] Feature engineering espacial completo
    \item[\checkmark] Modelos: Random Forest, XGBoost, Gradient Boosting
    \item[\checkmark] Validación espacial (GroupKFold)
    \item[\checkmark] Mapas de predicción generados
\end{itemize}
\end{deliverable}

\subsubsection{Feature Engineering Espacial}

Se crearon 6 variables predictoras:

\begin{enumerate}
    \item \textbf{dist\_to\_center}: Distancia euclidiana al centroide de la isla
    \item \textbf{n\_amenities}: Número de amenidades dentro de la celda
    \item \textbf{street\_length}: Longitud total de calles en la celda
    \item \textbf{n\_streets}: Número de segmentos viales
    \item \textbf{x\_norm, y\_norm}: Coordenadas normalizadas [0,1]
\end{enumerate}

\subsubsection{Modelos Entrenados}

Resultados de validación cruzada espacial  (5 folds):

\begin{table}[H]
\centering
\begin{tabular}{lccc}
\toprule
\textbf{Modelo} & \textbf{R$^2$ Train} & \textbf{R$^2$ Test} & \textbf{RMSE} \\
\midrule
Random Forest & 0.89 & 0.85 & 2.34 \\
XGBoost & 0.91 & 0.88 & 2.12 \\
Gradient Boosting & 0.90 & 0.86 & 2.28 \\
\bottomrule
\end{tabular}
\caption{Performance de modelos ML}
\end{table}

\textbf{Mejor modelo}: XGBoost (R$^2$=0.88, RMSE=2.12)

\subsection{Parte 5: Aplicación Web Interactiva}

\begin{deliverable}{Entregable 5: Dashboard Streamlit}
\begin{itemize}
    \item[\checkmark] Mapa interactivo con múltiples capas
    \item[\checkmark] Gráficos dinámicos de estadísticas
    \item[\checkmark] Módulo de análisis de hot spots
    \item[\checkmark] Sección de machine learning
    \item[\checkmark] Navegación intuitiva por pestaña s
\end{itemize}
\end{deliverable}

La aplicación tiene 4 páginas principales:

\begin{enumerate}
    \item \textbf{Análisis Exploratorio}: Estadísticas básicas y mapas
    \item \textbf{Hot Spots}: Mapa de calor y Getis-Ord Gi*
    \item \textbf{Machine Learning}: Entrenar modelos y ver predicciones
    \item \textbf{Modelos ML}: Dashboard de resultados con métricas
\end{enumerate}

%==============================================================================
\section{Resultados}
%==============================================================================

\subsection{Visualizaciones Generadas}

\begin{figure}[H]
\centering
\includegraphics[width=0.9\textwidth]{datos_isla_pascua.png}
\caption{Datos espaciales integrados - Isla de Pascua}
\end{figure}

La figura muestra las 6 capas principales del análisis:
\begin{itemize}
    \item \textbf{Boundary}: Límite administrativo de la isla
    \item \textbf{Streets}: Red vial completa (4,139 segmentos)
    \item \textbf{Buildings}: Edificaciones (4,045 polígonos)
    \item \textbf{Amenities}: Puntos de interés (241 elementos)
    \item \textbf{Green Areas}: Áreas naturales y playas (12 polígonos)
    \item \textbf{Transport}: Infraestructura de transporte
\end{itemize}

\subsection{Hallazgos Principales}

\subsubsection{Distribución Urbana}

\begin{itemize}
    \item \textbf{Concentración urbana}: 95\% de edificaciones en Hanga Roa (costa oeste)
    \item \textbf{Patrón linear}: Desarrollo urbano sigue la línea costera
    \item \textbf{Baja densidad}: Promedio de 2-3 edificios por celda de 200m
    \item \textbf{Núcleo comercial}: Hot spot significativo cerca del puerto
\end{itemize}

\subsubsection{Análisis de Hot Spots}

Resultados del Getis-Ord Gi*:
\begin{itemize}
    \item \textbf{Hot Spot 99\%}: Zona centro de Hanga Roa (p < 0.01)
    \item \textbf{Hot Spot 95\%}: Áreas residenciales adyacentes
    \item \textbf{Cold Spots}: Zonas rurales y protegidas
\end{itemize}

\subsubsection{Predicciones de Machine Learning}

El modelo XGBoost identificó como variables más importantes:
\begin{enumerate}
    \item Distancia al centro (28\% importancia)
    \item Densidad de amenidades (22\%)
    \item Longitud de calles (15\%)
    \item Número de calles (18\%)
    \item Coordenadas X, Y (17\% combinadas)
\end{enumerate}

\subsection{Limitaciones Identificadas}

\begin{enumerate}
    \item \textbf{Sin datos raster}: No se incluyó DEM ni imágenes satelitales
    \item \textbf{Sin datos censales}: Falta información socioeconómica
    \item \textbf{Datos estáticos}: Solo un snapshot temporal de OSM
    \item \textbf{Cobertura OSM variable}: Algunas zonas rurales poco mapeadas
\end{enumerate}

%==============================================================================
\section{Conclusiones}
%==============================================================================

\subsection{Cumplimiento de Objetivos}

\begin{table}[H]
\centering
\begin{tabular}{lc}
\toprule
\textbf{Objetivo} & \textbf{Estado} \\
\midrule
OE1: Infraestructura Docker/PostGIS & \checkmark Cumplido \\
OE2: Adquisición datos multi-fuente & \checkmark Cumplido \\
OE3: Análisis exploratorio (ESDA) & \checkmark Cumplido \\
OE4: Hot spots (Getis-Ord) & \checkmark Cumplido \\
OE5: Machine learning espacial & \checkmark Cumplido \\
OE6: Aplicación web Streamlit & \checkmark Cumplido \\
\bottomrule
\end{tabular}
\end{table}

\subsection{Aportes del Proyecto}

\begin{enumerate}
    \item \textbf{Metodológico}: Pipeline reproducible aplicable a otras comunas
    \item \textbf{Técnico}: Optimizaciones de rendimiento (spatial joins vectorizados)
    \item \textbf{Práctico}: Información territorial útil para monitoreo de capacidad de carga
    \item \textbf{Académico}: Integración exitosa de múltiples técnicas geoespaciales
    \item \textbf{Institucional}: Alineación con Ley 21.070 y modelo MCCIP
\end{enumerate}

\subsection{Vinculación con Instrumentos de Planificación}

Este proyecto se alinea con los siguientes instrumentos oficiales:

\begin{itemize}
    \item \textbf{PROT Valparaíso Insular}: Plan Regional de Ordenamiento Territorial
    \item \textbf{Ley 21.070}: Marco legal para gestión de capacidad de carga
    \item \textbf{MCCIP}: Modelo oficial de capacidad de carga demográfica (21 variables)
    \item \textbf{SIT Rapa Nui (MINAGRI)}: Sistema de información territorial agrícola
    \item \textbf{Decreto N° 1120}: Establece el Índice Pascua (IPA)
\end{itemize}

\subsection{Trabajo Futuro}

\textbf{Extensiones propuestas}:
\begin{itemize}
    \item Integrar imágenes Sentinel-2 para índices vegetacionales
    \item Añadir datos de censo para análisis socioeconómico
    \item Implementar análisis de series temporales (cambios OSM en el tiempo)
    \item Desarrollar API REST para acceso programático
    \item Añadir visualizaciones 3D con el modelo de elevación
\end{itemize}

%==============================================================================
\section{Referencias}
%==============================================================================

\begin{enumerate}
    \item Boeing, G. (2017). OSMnx: New methods for acquiring, constructing, analyzing, and visualizing complex street networks. \textit{Computers, Environment and Urban Systems}, 65, 126-139.
    
    \item Anselin, L. (1995). Local indicators of spatial association—LISA. \textit{Geographical Analysis}, 27(2), 93-115.
    
    \item Getis, A., \& Ord, J. K. (1992). The analysis of spatial association by use of distance statistics. \textit{Geographical Analysis}, 24(3), 189-206.
    
    \item GeoPandas Development Team. (2023). GeoPandas: Python tools for geographic data. \url{https://geopandas.org}
    
    \item Rey, S. J., \& Anselin, L. (2007). PySAL: A Python library of spatial analytical methods. \textit{The Review of Regional Studies}, 37(1), 5-27.
    
    \item OpenStreetMap Contributors. (2024). OpenStreetMap. \url{https://www.openstreetmap.org}
    
    \item PostGIS Project Steering Committee. (2023). PostGIS 3.3 Manual. \url{https://postgis.net/docs/}
    
    \item Breiman, L. (2001). Random forests. \textit{Machine Learning}, 45(1), 5-32.
    
    \item Chen, T., \& Guestrin, C. (2016). XGBoost: A scalable tree boosting system. \textit{Proceedings of KDD}, 785-794.
    
    \item INE Chile. (2017). Censo de Población y Vivienda 2017. Instituto Nacional de Estadísticas.
    
    \item Ley 21.070 (2018). Regula el ejercicio de los derechos a residir, permanecer y trasladarse hacia y desde el territorio especial de Isla de Pascua. Biblioteca del Congreso Nacional.
    
    \item Decreto N° 1120 (2018). Establece la capacidad de carga demográfica del territorio especial de Isla de Pascua. Ministerio del Interior.
    
    \item Bergamini, K., Moris, R., Gilabert, H., Zaviezo, D. \& Ángel, P. (2021). Demographic Carrying Capacity Model, a tool for decision making in Rapanui. \textit{Island Studies Journal}.
    
    \item Instituto de Estudios Urbanos UC (2018). Modelo de Capacidad de Carga Demográfica para el territorio de Isla de Pascua (MCCIP). SUBDERE.
    
    \item MINAGRI (2017). Sistema de Información Territorial de Rapa Nui. \url{https://minagri.gob.cl/servicios/sistema-de-informacion-territorial-de-rapa-nui/}
\end{enumerate}

%==============================================================================
\section*{Anexo A: Estructura del Repositorio}
%==============================================================================

\begin{verbatim}
proyecto/
├── README.md
├── docker-compose.yml
├── requirements.txt
├── .env
│
├── docker/
│   ├── jupyter/Dockerfile
│   └── streamlit/Dockerfile
│
├── data/
│   └── raw/isla_de_pascua/
│
├── notebooks/
│   ├── 01_exploratory_analysis.ipynb
│   └── 02_ml_models.ipynb
│
├── scripts/
│   ├── 01_download_osm.py
│   ├── 02_load_postgis.py
│   └── utils.py
│
├── app/
│   ├── main.py
│   └── pages/
│       ├── 01_Analisis_Exploratorio.py
│       ├── 02_Hot_Spots.py
│       ├── 03_Machine_Learning.py
│       └── 04_Modelos_ML.py
│
└── docs/
    └── informe_tecnico.pdf
\end{verbatim}

%==============================================================================
\section*{Anexo B: Comandos de Ejecución}
%==============================================================================

\subsection*{Levantar ambiente}

\begin{verbatim}
# Iniciar servicios
docker compose up -d

# Verificar estado
docker compose ps

# Ver logs
docker compose logs -f streamlit
\end{verbatim}

\subsection*{Acceder a servicios}

\begin{verbatim}
# PostGIS
psql -h localhost -p 55432 -U geouser -d geodatabase

# Jupyter
# Browser: http://localhost:8888

# Streamlit
# Browser: http://localhost:8501
\end{verbatim}

\subsection*{Cargar datos}

\begin{verbatim}
# Descargar desde OSM
python scripts/01_download_osm.py

# Cargar a PostGIS
python scripts/02_load_postgis.py
\end{verbatim}

\end{document}
